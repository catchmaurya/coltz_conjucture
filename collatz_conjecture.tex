
\documentclass{article}
\usepackage{amsmath}
\usepackage{amsfonts}
\usepackage{geometry}
\geometry{margin=1in}
\title{The Collatz Conjecture: Structured Mathematical and Computational Study}
\author{Maurya Allimuthu\\ \texttt{catchmaurya@gmail.com} }
\date{\today}

\begin{document}

\maketitle

\section{Problem Statement}

The \textbf{Collatz Conjecture} (also known as the 3n + 1 problem) posits:

\textit{For any positive integer \( n \), repeated application of the rule}
\[
T(n) = \begin{cases}
\frac{n}{2}, & \text{if } n \equiv 0 \ (\mathrm{mod}\ 2) \\
3n + 1, & \text{if } n \equiv 1 \ (\mathrm{mod}\ 2)
\end{cases}
\]
\textit{will eventually reach 1.}

Despite its simple definition, no general proof or counterexample exists.

\section{Analysis and Observations}

\subsection{Symbolic Equation Tracking}

For small integers, symbolic manipulation reveals the sequence:

\begin{itemize}
    \item \( n = 8 \):
    \[
    8 \rightarrow 4 \rightarrow 2 \rightarrow 1 \Rightarrow 8 \cdot \left(\frac{1}{2}\right)^3 = 1
    \]
    \item \( n = 3 \):
    \[
    3 \rightarrow 10 \rightarrow 5 \rightarrow 16 \rightarrow 8 \rightarrow \dots \rightarrow 1
    \]
    Symbolically:
    \[
    (3 \cdot ((3n + 1)/2) + 1) \cdot \left(\frac{1}{2}\right)^4 = 1 \Rightarrow n = 3
    \]
\end{itemize}

\subsection{Backward Construction Strategy}

We define \( T^{-1}(n) \) as:
\begin{itemize}
    \item If \( n \) is even, then \( T^{-1}(n) = 2n \)
    \item If \( n \equiv 4 \ (\mathrm{mod}\ 6) \), then \( T^{-1}(n) = \frac{n - 1}{3} \)
\end{itemize}

This leads to a reverse tree structure rooted at 1, where branches grow from known Collatz paths backward.

\section{Approaches to the Problem}

\subsection{Dynamic Programming / Memoization}

We compute and store the full Collatz paths for all \( n \leq 1000 \). If any number \( x > 1000 \) hits a known value during its path, we terminate the computation early.

Let \( S = \{1, 2, \dots, 1000\} \). If \( T_i(x) \in S \), then \( T(x) \to 1 \) by inheritance.

\subsection{Recursive Block Reduction}

If the conjecture holds for \( [1, 1000] \), then:
\begin{itemize}
    \item \( [1001, 2000] \): even numbers halve into \( \leq 1000 \)
    \item \( [1001, 3001] \): many odd values produce \( \leq 1000 \) via \( (3n + 1)/2 \)
\end{itemize}

Thus, the original range \( [1, 1000] \) acts as a proof base, extending recursively over wider domains.

\section{Proof Sketch}

\begin{itemize}
    \item Symbolic expressions for number paths
    \item Reverse tree modeling using \( T^{-1}(n) \)
    \item Cached values for efficient verification
    \item Block-wise reduction pattern
\end{itemize}

\section{Graph / Tree Visualization}

A tree structure rooted at 1 can be constructed:
\begin{itemize}
    \item Nodes represent values \( n \)
    \item Edges represent Collatz transitions: \( n \rightarrow T(n) \)
    \item Backward expansion uses \( T^{-1}(n) \) rules
\end{itemize}

Graph implementation can be performed using \texttt{networkx} and \texttt{matplotlib}.

\section{Code Implementation}

Python code computes paths and step counts for \( n \in [1, 1000] \) with memoization. Example snippet:
\begin{verbatim}
collatz_cache = {}

def collatz_path(n):
    if n in collatz_cache:
        return collatz_cache[n]
    path = []
    original_n = n
    while n != 1:
        path.append(n)
        if n % 2 == 0:
            n = n // 2
        else:
            n = 3 * n + 1
    path.append(1)
    collatz_cache[original_n] = path
    return path
\end{verbatim}

\section{Symbolic Equation Generator (Planned)}

A tool can be developed to:
\begin{itemize}
    \item Record each transformation symbolically
    \item Build nested expressions of the form:
    \[
    n \rightarrow \frac{3n + 1}{2} \rightarrow \dots \rightarrow 1
    \]
\end{itemize}

\section{Conclusion}

We combine:
\begin{itemize}
    \item Symbolic tracking of number reductions
    \item Reverse tree modeling (\( T^{-1}(n) \))
    \item Dynamic programming for optimization
    \item Block-based recurrence strategy
\end{itemize}

This multi-layered approach blends theory and computation to progress toward a structured proof or reduction strategy for the Collatz Conjecture.

\end{document}
